\section{Text Justification}
Text justification refers to a form of typographic  alignment where the spaces between words are stretched or compressed to align both the left and right ends of each line of text. The text justification problem is defined as follows: given a sequence of words, and a limit on the number of characters that can be put in one line (line width). Put line breaks in the given sequence such that the lines are printed neatly. This problem aims to produce balanced lines of text, not having too much or too little extra space.

The application of this problem is within word processing software where text justification is an option of typographic alignment. It is often preferred over standard text alignment for documents with a large amount of text with not so many pictures or similar elements which would break the text up. So text justification will make the document appear neater and in turn easier to read. 


\section{Subset Sum}
The subset sum problem is as follows: given a set of integers and a value \textit{sum}, determine whether there is a subset of the given set with a sum equal to \textit{sum}.

One application of the subset sum problem is in the field of cryptography, specifically with computer passwords. Instead of storing the user's password in plain-text in the internal files, the computer generates a large set of numbers. A password is a subset of the full set, so instead of having the password for the user, the computer keeps the total associated with the appropriate subset. When the user types in the subset, the computer tests whether the total is correct. It does not keep a record of the subset. Thus impersonation is possible only if somebody can reconstruct the subset knowing the total and numbers within\cite{passwords}.

A second application is with message verification. A sender (\textit{S}) wants to send messages to a receiver \textit{R}. Keeping the message secret is not important. However, \textit{R} wants to be sure that the message he is receiving is not from an imposter and has not been tampered with. \textit{S} and \textit{R} agree on a main set of a certain size as well as a set of totals. These numbers may be publicly known, but only \textit{S} knows which subsets of the main set correspond to which total. The message sent by \textit{S} is actually a number of subsets of the main set corresponding to the message he wants to send\cite{passwords}.


\section{Word Break}
Given a dictionary of words and a string of characters, find out if the string of characters can be broken into individual valid words from the dictionary.


