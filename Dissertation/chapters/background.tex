\section{History of Dynamic Programming}
Initially the term \textit{Dynamic Programming} was used by Richard Bellman, a mathematician, in the 1940s to describe the process of solving problems where one needs to find the best decisions one after another. He would later refine his definition in 1953 to refer to nesting smaller decision problems inside of larger problems, which is what remains the modern definition \cite{bellman}. The word \textit{dynamic} was chosen by Bellman due to the time-varying aspect of the problems and because it sounded impressive \cite{dynamic}. \textit{Programming} was chosen as to refer to the use of the method to find an optimal program and as a synonym to the word planning. Upon establishment of this topic, dynamic programming began to grow and became recognized as a substantial area of interest within the field of computer science. As the field of Computer Science expanded over the years, dynamic programming remains a relevant topic with a multitude of algorithms making use of the method.

\section{Dynamic Programming Overview}
Dynamic programming is both a mathematical optimization method and a computer programming method. In both contexts it refers to simplifying a complicated problem by breaking it down into simpler subproblems in a recursive manner. These subproblems are each solved just once and their solutions are stored - ideally, using a memory-based data structure. The next time the same subproblem occurs, instead of recomputing its solution, the previously computed solution is looked up, thus saving computation time at the expense of some expenditure in storage space\cite{dynamic-wiki}. Dynamic programming is applicable to problems exhibiting two specific properties: optimal substructure and overlapping subproblems.
\smallbreak
A problem is said to have optimal substructure if an optimal solution can be constructed efficiently from optimal solutions of its subproblems\cite{design-manual}. While a problem is said to have overlapping subproblems if the problem can be broken down into subproblems which are reused several times or a recursive algorithm for the problem solves the same subproblem over and over rather than always generating new subproblems\cite{IntroAlgo}. If a problem has both of these properties then a dynamic programming approach to problem solving is applicable. 

\section{Related Work}
As part of the undertaking of this project, I made sure to research the topic of dynamic programming in relative depth in order to gain a proper understanding of the topic. In searching for books and research papers covering dynamic programming, I identified several which I felt were worthwhile and helpful to use as reference points due to a close relation to the aspects of dynamic programming that I wish to study. Firstly, \textit{Dynamic Programming in Computer Science}\cite{DPiCS} is a 1979 paper written by Kevin Q. Brown which somewhat generally introduces the topic of dynamic programming in a relatively easy to understand fashion. It also provides detailed examples of dynamic programming problems and algorithms and the theory behind them. I found this paper provided me with a good introduction to the topic and it was useful to return to later once I had learned some of the greater context regarding dynamic programming, given the detailed documentation. Even if it displayed the topic more mathematically complex than I wished to go for my project.
\smallbreak

Another paper I red through was \textit{Learning to act using real-time dynamic programming} \cite{learning2act}, written by Andrew G. Barto, Steven J. Bradtke and Satinder P Singh. This paper primarily looks at the use of dynamic programming in artificial intelligence. I found that its coverage of a variety of dynamic programming problems and general theory were useful in understanding the greater application of the dynamic programming method. The paper presents and documents a variety of problems that mainly focus on the learning capabilities through memoization techniques. While my approach to this topic is more general and implementation focused, this paper provided useful information on the topic and was helpful in understanding the variety of applications that dynamic programming allows for. 
\smallbreak

The book \textit{Introduction to Algorithms} by Thomas H. Cormen, Charles E. Leiserson, Ronald L. Rivest and Clifford Stein has a chapter devoted to dynamic programming \cite{IntroAlgo}. Chapter 15 of the book covers four dynamic programming problems at length, including the Longest Common Subsequence problem, which I chose to be one of the first problems to look into and implement myself. This chapter of the book also explores the dynamic programming method with a section focusing in on the two key concepts within the topic. This chapter of the book was very useful for me as a concise introduction to the topic as well as providing some inspiration regarding how I would present my own work and approach the project as a whole.
\smallbreak

Similarly another book, \textit{The Algorithm Design Manual} by Steven S Skiena, has a chapter devoted to dynamic programming \cite{design-manual}. In chapter 8 of the book, a selection of dynamic programming problems are discussed in depth as well as providing code and some diagrams for the various problems. The book also observes links between different problems and how the understanding of one allows for easier understanding of a similar related problem. Similar to \textit{Introduction to Algorithms}, this book was very helpful in providing an idea of how I should present my work and the sorts of research and documentation I'd have to delve into for each dynamic programming problem I would tackle. It also covered certain problems I chose to implement myself, namely the edit distance and longest increasing subsequence problems.
\smallbreak

One last book that I read through to assist in working on this project was \textit{Dynamic Programming For Coding Interviews} which is written by Meenakshi \& Kamal Rawat \cite{interviews}. Unlike the previous books and research papers mentioned, I read this book later in my project development time line, when I had already began properly programming. This book was very helpful in re-enforcing the key concepts of dynamic programming in my mind as well as how to approach the problem solving method. Mostly, this book goes into detail on the method of dynamic programming for problem solving, while also containing a lot of useful examples, including a few problems that I implemented for my project; the minimum coin change and longest palindromic subsequence problems.
