\section{Motivation}

% Initial Intro
One of the key features in the field of computer science is the constant strive for improvement. Whether it be a piece of hardware, software or an algorithm, it is ideal to continually search for new ways to improve on a variety of aspects such as usability, structuring and, perhaps most importantly; performance. Dynamic programming is a method that captures this constant push for faster, more efficient problem solving. As a model of programming, it holds the core purpose of breaking existing recursive problems down into smaller subproblems to allow for greater efficiency in the solving process as well as in general optimization.
\smallbreak

% What I'm doing with the project - Core of the diss
This project looks at the dynamic programming method of implementation and how it can be applied to ten computational problems to improve upon their base method and increase the efficiency of solving. From this a clear understanding can be obtained of the advantageous elements of implementing a dynamic programming approach to improve upon existing solutions to problems. For each problem the premise and strategy for solving will be explored through documentation. This documentation introduces the premise of the problem in detail as well as providing written examples to demonstrate this in a clear, understandable form. Following this, the approach to writing an algorithm to solve the problem is discussed and demonstrated in a fashion meant to be easily understood by somebody with limited dynamic programming knowledge, while also retaining usefulness for those who are knowledgeable in that field. Additionally the real world applications of these problems will be explored, in order show just how wide the usage of dynamic programming techniques can be spread. 
\smallbreak

% Java programs intro
Alongside this , each problem tackled in this project has an accompanying Java program which exhibits the dynamic programming method of solving that particular problem. These are intended to provide greater understanding of the computational process behind each problem and how dynamic programming assists in this way. Each Java program will show how the core dynamic programming principles can be applied to a more specific problem, meaning links between the programs can be highlighted as well as more unique applications of the technique.\\


Below is a full list of the dynamic programming problems that are explored through this project:

\begin{table}[h]
	\resizebox{\textwidth}{!}{%
		\begin{tabular}{ll}
			• Minimum Coin Change Problem & • Longest Palindromic Substring Problem \\
			• Longest Increasing Subsequence Problem & • Maximum Square Sub-Matrix with all 1's Problem \\
			• Longest Common Subsequence Problem & • Subset Sum Problem \\
			• Edit Distance Problem & • Text Justification Problem \\
			• String Interleaving Problem & • Word Break Problem
		\end{tabular}%
	}
\end{table}

% Project Uses & achievements
This project aims to be a concise educational resource for the topic of dynamic programming, mainly covering how it can be implemented in the process of problem solving. If somebody is unaware of dynamic programming and requires an idea of what the concept is and how it can be easily applied to a set of problems, this project will be able to provide assistance and relative guidance in how to approach a dynamic programming solution. The problems explored can also more explicitly provide insight into specific solutions to problems, given that the Java code is referenced as well as their common applications.
\smallbreak

% Achievements 
Through this project I have achieved what I set out to do, which was to satisfyingly document and program ten dynamic programming problems which I personally found to be interesting to research and understand. Adding to this, through the development of Java programs for each problem, further educational merit can be obtained as well as achieving fully-functional implementation of ten interesting algorithm focused pieces of code.
\smallbreak

\par\noindent
% Aims
To summarise, this projects three key aims are to:
\begin{itemize}
	\item Understand the benefits of a dynamic programming approach to algorithmic problem-solving
	\item Document and achieve understanding of ten different dynamic programming problems
	\item Implement each of the ten problems into functioning Java programs
\end{itemize}


\section{Document Outline}
